% This work is licensed under a Creative Commons CC BY 4.0 License.
\section{Speicherverwaltung mit Buddy-Systemen}

Zugeh"orige Folien:  VII-22--27 \\

Gegeben sei ein Rechner mit 32 MByte Hauptspeicher.  Die
Speicherverwaltung soll mittels eines gewichteten Buddy-Systems
erfolgen.  Es sollen nun nacheinander Speicherbereiche mit den
folgenden Eigenschaften angefordert / freigegeben werden:

\begin{center}
  \begin{tabular}{clcc}
    \toprule
    Nr. & Operation   & Name & Gr"o"se [MByte] \\
    \midrule
    1   & Anforderung & A    & 13 \\
    2   & Anforderung & B    & 5  \\
    3   & Freigabe    & A    &    \\
    4   & Anforderung & C    & 1  \\
    5   & Anforderung & D    & 2  \\
    \bottomrule
  \end{tabular}
\end{center}

Stellen Sie die Speicherbelegung nach jeder Anforderung als Bin"arbaum
dar.  Geben Sie weiterhin an, was die gr"o"stm"ogliche
Speicheranforderung ist, welche nach Ausf"uhrung der Anfragen noch
erfolgreich ausgef"uhrt werden kann, und viel Speicher durch interne
Fragmentierung bedingt nicht mehr nutzbar ist.  Tun Sie dies f"ur

\begin{enumerate}
\item[(a)] ein ungewichtetes Buddy-System.
  
\item[(b)] ein gewichtetes Buddy-System.  Dabei soll ein Knoten, falls
  seine Gr"o"se ohne Rest durch 3 teilbar ist, im Verh"altnis 1:2
  aufgeteilt werden.  Ist seine Gr"o"se restlos durch 4, aber nicht
  durch drei teilbar, so wird kann er in zwei Speicherbereiche mit dem
  Verh"altnis 1:3 aufgeteilt werden.
\end{enumerate}
