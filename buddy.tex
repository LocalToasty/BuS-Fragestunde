\section{Speicherverwaltung mit gewichteten Buddy-Systemen}

Zugeh"orige Folien:  VII-22--27 \\

Gegeben sei ein Rechner mit 32 MByte Hauptspeicher.  Die
Speicherbelegung soll mittels eines gewichteten Buddy-Systems
erfolgen.  Dabei soll ein Knoten, falls seine Gr"o"se ohne Rest durch
3 teilbar ist, im Verh"altnis 1:2 aufgeteilt werden.  Ist seine
Gr"o"se restlos durch 4, aber nicht durch drei teilbar, so wird kann
er in zwei Speicherbereiche mit dem Verh"altnis 1:3 aufgeteilt werden.

Es sollen nun nacheinander Speicherbereiche mit den folgenden
Eigenschaften angefordert / freigegeben werden:

\begin{center}
  \begin{tabular}{|c|l|c|c|}
    \hline
    Nr. & Operation   & Name & Gr"o"se [MByte] \\
    \hline \hline
    1   & Anforderung & A    & 13 \\
    2   & Anforderung & B    & 5  \\
    3   & Freigabe    & A    &    \\
    4   & Anforderung & C    & 1  \\
    5   & Anforderung & D    & 2  \\
    \hline
  \end{tabular}
\end{center}

Stellen Sie die Speicherbelegung nach jeder Anforderung als Bin"arbaum
dar.  Geben Sie weiterhin an, was die gr"o"stm"ogliche
Speicheranforderung ist, welche nach Ausf"uhrung der Anfragen noch
erfolgreich ausgef"uhrt werden kann, und viel Speicher durch interne
Fragmentierung bedingt nicht mehr nutzbar ist.
