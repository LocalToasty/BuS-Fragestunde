\paragraph{L"osung}
\begin{enumerate}
  \item Anforderung A (13 MByte):

  \Tree[.32
    8
    [.24
      8
      {A 13/16}
    ]
  ]

  \item Anforderung B (5 MByte):

  \Tree[.32
    [.8
      2
      {B 5/6}
    ]
    [.24
      8
      {A 13/16}
    ]
  ]

  \item Freigabe A:

  \Tree[.32
    [.8
      2
      {B 5/6}
    ]
    24
  ]

  \item Anforderung C (1 MByte):

  \Tree[.32
    [.8
      2
      {B 5/6}
    ]
    [.24
      [.8
        2
        [.6
          2
          [.4
            {C 1/1}
            3
          ]
        ]
      ]
      16
    ]
  ]

  \item Anforderung D (2 MByte):

  \Tree[.32
    [.8
      {D 2/2}
      {B 5/6}
    ]
    [.24
      [.8
        2
        [.6
          2
          [.4
            {C 1/1}
            3
          ]
        ]
      ]
      16
    ]
  ]
\end{enumerate}

Die gr"o"ste erf"ullbare Speicheranforderung ist gleich der Kapazit"at
des gr"o"sten, noch unbelegten Knotens, in diesem Fall also 16 MByte.
Insgesamt geht 1 MByte ($\approx 3\%$ des Gesamtspeichers) durch
interne Fragmentierung verloren.
