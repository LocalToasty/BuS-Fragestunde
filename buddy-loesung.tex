\paragraph{L"osung}
\begin{enumerate}[label=(\alph*)]
\item
  \begin{enumerate}[label=\arabic*.]
  \item Anforderung A (13 MByte):

    \Tree[.32
      {A 13/16}
      16
    ]

  \item Anforderung B (5 MByte):

    \Tree[.32
      {A 13/16}
      [.16
        {B 5/8}
        8
      ]
    ]

  \item Freigabe A:

    \Tree[.32
      16
      [.16
        {B 5/8}
        8
      ]
    ]

  \item Anforderung C (1 MByte):

    \Tree[.32
      16
      [.16
        {B 5/8}
        [.8
          [.4
            [.2
              {C 1/1}
              1
            ]
            2
          ]
          4
        ]
      ]
    ]

  \item Anforderung D (1 MByte):

    \Tree[.32
      16
      [.16
        {B 5/8}
        [.8
          [.4
            [.2
              {C 1/1}
              1
            ]
            {D 2/2}
          ]
          4
        ]
      ]
    ]

  \end{enumerate}

  Die gr"o"ste erfolgreich ausf"uhrbare Speicharanforderung betr"agt
  16 MByte.  Insgesamt gehen 3 MByte Speicher ($\approx 9\%$) durch
  interne Fragmentierung verloren.

\item
  \begin{enumerate}[label=\arabic*.]
  \item Anforderung A (13 MByte):

    \Tree[.32
      8
      [.24
        8
        {A 13/16}
      ]
    ]

  \item Anforderung B (5 MByte):

    \Tree[.32
      [.8
        2
        {B 5/6}
      ]
      [.24
        8
        {A 13/16}
      ]
    ]

  \item Freigabe A:

    \Tree[.32
      [.8
        2
        {B 5/6}
      ]
      24
    ]

  \item Anforderung C (1 MByte):

    \Tree[.32
      [.8
        2
        {B 5/6}
      ]
      [.24
        [.8
          2
          [.6
            2
            [.4
              {C 1/1}
              3
            ]
          ]
        ]
        16
      ]
    ]

  \item Anforderung D (2 MByte):

    \Tree[.32
      [.8
        {D 2/2}
        {B 5/6}
      ]
      [.24
        [.8
          2
          [.6
            2
            [.4
              {C 1/1}
              3
            ]
          ]
        ]
        16
      ]
    ]
  \end{enumerate}

  Die gr"o"ste erf"ullbare Speicheranforderung ist gleich der Kapazit"at
  des gr"o"sten, noch unbelegten Knotens, in diesem Fall also 16 MByte.
  Insgesamt geht 1 MByte ($\approx 3\%$ des Gesamtspeichers) durch
  interne Fragmentierung verloren.
\end{enumerate}
