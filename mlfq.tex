% This work is licensed under a Creative Commons CC BY 4.0 License.
\section{Multilevel Feedback Queueing}

Zugeh"orige Folien:  V-31--33 \\

Gegeben sei ein Multilevel Feedback Queueing mit vier
Priorit"atsklassen, wobei jede Klasse eine eigene Warteschlange,
Bedienstrategie sowie ein eigenes Quantum hat.  Dabei ist Klasse 1
diejenige mit der h"ochsten Priorit"at:

\begin{center}
  \begin{tabular}{ccc}
    \toprule
    Klasse & Quantum  & Bedienstrategie \\
    \midrule
    1      & 2        & SRPT \\
    2      & 4        & LIFO \\
    3      & 8        & RR${}_2$ \\
    4      & $\infty$ & FIFO \\
    \bottomrule
  \end{tabular}
\end{center}

Ein eingehender Prozess wird in der Zeiteinheit nach seinem Eintreffen
in die seiner Priorit"atsklasse zugeh"orige Warteschlange eingereiht.
Wann immer ein Prozess sein Quantum aufgebraucht hat, wird er an das
Ende der Warteschlange mit n"achstniedrigerer Priorit"at angeh"angt.
Wird ein Prozess unterbrochen, bevor er sein Zeitquantum aufgebraucht
hat (weil zum Beispiel ein neuer Prozess mit h"oherer Priorit"at
eintrifft), so wird er an den Anfang seiner gegenw"artigen
Warteschlange gestellt, darf aber, wenn er das n"achste mal aufgerufen
wird, nur sein Restquantum aufbrauchen.

Nacheinander kommen verschiedene Prozesse mit folgenden Priorit"aten und
Bedienzeiten an:

\begin{center}
  \begin{tabular}{cccc}
    \toprule
    Prozess & Ankunftszeit & Bedienzeit & Priorit"at \\
    \midrule
    A       & 1            & 3          & 3 \\
    B       & 1            & 6          & 1 \\
    C       & 3            & 3          & 1 \\
    D       & 6            & 9          & 4 \\
    E       & 8            & 2          & 2 \\
    \bottomrule
  \end{tabular}
\end{center}

Ermitteln Sie f"ur jeden Zeitschritt $t \in \{1, \dots, 20\}$ welcher
Prozess gerade l"auft, und welche Prozesse sich in welcher
Priorit"atsklasse befinden.  Stellen Sie die Ergebnisse tabellarisch in
folgender Form dar:

\begin{center}
  \begin{tabular}{rlllllc}
    \toprule
    t  & 1. SRPT (2) & 2. FILO (4)      & 3. RR${}_2$ (8)    & 4. FIFO ($\infty$) & Incoming     & Running \\
    \midrule
    1  &             &                  &                    &                    & $A(3), B(7)$ & \\
    2  & $B_2(7)$    &                  & $A_8^2(3)$         &                    &              & $B(7)$ \\
    $\vdots$ & \multicolumn{5}{c}{$\vdots$} & $\vdots$ \\
    \bottomrule
  \end{tabular}
\end{center}
