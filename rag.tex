% This work is licensed under a Creative Commons CC BY 4.0 License.
\section{Deadlocks}

Zugeh"orige Folien: VI-106--109 \\

Auf einem System laufen drei Prozesse $P_1,P_2,P_3$, welche auf die vier
Betriebsmittel $A,B,C,D$ zugreifen.  Die Prozesse werden simultan
ausgef"uhrt.  Die Anforderungsmuster sehen dabei wie folgt aus:

\begin{center}
  \begin{tabular}{cccc}
    \toprule
    $t$ & $P_1$ & $P_2$ & $P_3$ \\
    \midrule
    1   & $C,4$ &       &       \\
    2   &       & $D,5$ & $B,3$ \\
    3   &       & $A,3$ &       \\
    4   &       &       & $C,4$ \\
    5   & $A,2$ & $B,1$ &       \\
    \bottomrule
  \end{tabular}
\end{center}

Dabei bedeutet ein Eintrag $B,n$, dass der Prozess zu der gegebenen
Zeiteinheit das Betriebsmittel $B$ f"ur $n$ Zeiteinheiten anfragt.

Von jedem Betriebsmittel steht genau ein Exemplar zur Verf"ugung.  Wenn
ein Prozess zum Zeitpunkt $t$ ein freies Betriebsmittel anfordert, so
wird dieses dem Prozess zum Zeitpunkt $t$ zugeteilt, kann im Zeitraum $t
+ 1$ bis $t + a$ verwendet werden und kann zum Zeitpunkt $t + a + 1$
wieder einem anderen Prozess zugeteilt werden. Ist das Betriebsmittel
dagegen schon in Benutzung, wird der Prozess so lange suspendiert, bis
es wieder freigegeben wird.

Konstruieren Sie den Resource-Allocation-Graph zum Zeitpunkt $t=5$ und
entscheiden Sie begr"undet, ob ein Deadlock vorliegt.
