% This work is licensed under a Creative Commons CC BY 4.0 License.
\paragraph{L"osung}

Man betrachte f"ur die Zeiten $t \in \{1, \dots, 5\}$, welcher Prozess
welche Betriebsmittel benutzt:

\begin{center}
  \begin{tabular}{cccccccc}
    \toprule
    $t$ & $P_1,C$ & $P_2,D$ & $P_3,B$ & $P_2,A$ & $P_3,C$ & $P_1,A$ & $P_2,B$ \\
    \midrule
    1   & $Z,4$   &         &         &         &         &         &         \\
    2   & $R,4$   & $Z,5$   &  $Z,3$  &         &         &         &         \\
    3   & $R,3$   & $R,5$   &  $R,3$  & $Z,3$   &         &         &         \\
    4   & $R,2$   & $R,4$   &  $*,2$  & $R,3$   & $W,4$   &         &         \\
    5   & $*,1$   & $*,3$   &  $*,2$  & $*,2$   & $W,4$   & $W,2$   & $W,1$   \\
    \bottomrule
  \end{tabular}

$ $\\
W: Waiting, Z: Zugeteilt, R: Running, *: Blockiert \\
\end{center}

Zum Zeitpunkt $t = 5$ ergibt sich somit folgender Resource Allocation
Graph:

\begin{center}
  \begin{tikzpicture}
    \node (p1)[process]  at (1,2)   {$P_1$};
    \node (p2)[process]  at (3.5,1) {$P_2$};
    \node (p3)[process]  at (1,0)   {$P_3$};

    \node (a) [resource] at (2.5,2) {$A$};
    \node (b) [resource] at (2.5,0) {$B$};
    \node (c) [resource] at (0,1)   {$C$};
    \node (d) [resource] at (5,1)   {$D$};

    \draw[allocated] (p1) -> (a);
    \draw[allocated] (a)  -> (p2);
    \draw[allocated] (p2) -> (b);
    \draw[allocated] (b)  -> (p3);
    \draw[allocated] (p3) -> (c);
    \draw[allocated] (c)  -> (p1);
    \draw[allocated] (d)  -> (p2);
  \end{tikzpicture}
\end{center}

Wie zu sehen ist, existiert im RAG ein Zykel, weswegen eine der
Deadlock-Bedingungen, Circular Wait, erf"ullt ist.  Weiterhin war
vorgegeben, dass jedes der Betriebsmittel nur einfach zur Verf"ugung
steht, also ist auch Exclusive Use erf"ullt.  Da alle Prozesse ihre
Betriebsmittel halten, bis ihre Benutzung abgeschlossen ist, gilt Hold
and Wait.  Unter der Annahme, dass das Betriebssystem nicht-pr"aemptiv
ist (d.h. keinen Prozess terminiert oder ihm die Betriebsmittel
entzieht), liegt hier also ein Deadlock vor, da in diesem Fall alle
vier Deadlock-Bedingungen erf"ullt sind.
