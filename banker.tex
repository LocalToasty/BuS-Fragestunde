\section{Banker-Algorithmus}

Zugeh"orige Folien: VI-119--122 \\

Es existieren drei Prozesse $P_1, P_2, P_3$, welche die Betriebsmittel
$BM_1, BM_2, BM_3$ benutzen wollen.  Zu Anfang seinen $V(0) = (V_1(0),
V_2(0), V_3(0)) = (2, 5, 8)$ der jeweiligen Betriebsmittel frei
verf"ugbar.  Die Prozesse stellen nun folgende Anfragen an das
Betriebssystem, w"ahrend sie gleichzeitig die gegebene Anzahl an
Betriebsmitteln halten:

\begin{enumerate}
\item $Q_1^\text{max}(0) = (3, 3, 4), H_1(0) = (10, 0, 3)$
\item $Q_2^\text{max}(0) = (1, 2, 6), H_2(0) = (4, 3, 0)$
\item $Q_3^\text{max}(0) = (7, 1, 1), H_3(0) = (1, 4, 2)$
\end{enumerate}

Ermitteln Sie mit Hilfe des Banker-Algorithmus, ob diese Anfragen
sicher gew"ahrt werden k"onnen.
